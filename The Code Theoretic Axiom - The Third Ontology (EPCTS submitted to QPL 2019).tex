\documentclass[submission,copyright,creativecommons]{eptcs}
\providecommand{\event}{QPL 2019} % Name of the event you are submitting to
\usepackage{breakurl}             % Not needed if you use pdflatex only.
\usepackage{graphicx}
\title{The Code Theoretic Axiom - The Third Ontology}
\author{Klee Irwin
\institute{Quantum Gravity Research\\ Los Angeles, USA}
%\institute{School of Computer Science and Engineering\\
%University of New South Wales\thanks{A fine university.}\\
%Sydney, Australia}
\email{Klee@QuantumGravityResearch.org}
}

\def\titlerunning{The Code Theoretic Axiom}
\def\authorrunning{K. Irwin}
\begin{document}
\maketitle

\begin{abstract}
A logical physical ontology is code theory, wherein reality is neither deterministic nor random. In light of Conway and Kochen’s free will theorem \cite{conway2006free} and strong free will theorem \cite{conway2009strong}, we discuss the plausibility of a third axiomatic option - geometric language; the \textit{code theoretic axiom}. We suggest freewill choices at the syntactically free steps of a geometric language of spacetime form the code theoretic substrate upon which particle and gravitational physics emerge. 
\end{abstract}


\section{Introduction}
\label{intro}

\textbf{The Code Theoretic Axiom: }Reality is neither deterministic nor random. Instead, it is code-theoretic, wherein spacetime and particle therein are discrete and built of a Planck scale geometric code -- a finite set of shape-symbols, ordering rules and non-deterministic syntactical freedom.
\\

Broadly speaking, there are three axioms for a physical ontology one can assume. One is the idea that the universe is a deterministic causal chain or algorithm playing itself out. An example of this is the model of the Newtonian clockwork universe \cite{levinson2012clockwork}, which postulates that, if one knew the starting conditions, a powerful computer could predict every event \cite{laplace2012pierre}. A second option is the axiom of pure randomness, where a particle can appear anywhere in space and time according to probabilities dictated by quantum mechanics \cite{zettili2009quantum}. The third possibility is what we will henceforth refer to as code theory, where, for example, the Planck scale fabric of reality operates according to a geometric language with syntactical freedom creating order and preventing the existence of particles at certain spatiotemporal coordinates. Today, deterministic models are widely believed to be false \cite{bourget2014philosophers}, while the axiom of randomness is generally presumed to be true. This virtual consensus is due to two ideas. The first is the vastly popular Copenhagen interpretation of quantum mechanics \cite{griffiths2005introduction}, which stipulates that the universe is fundamentally random. The second is the widely accepted opinion that consciousness and freewill are real. 

The code theoretic axiom is a logical alternative to the two older ideas of determinism and pure randomness. Reality would be non-deterministic, not because it is random, but because it is a code -- a finite set of irreducible symbols and syntactical rules. Herein, we adopt the popular and reasoned view that freewill is real. Accordingly, we will not focus on deterministic models but instead consider the code theoretic and randomness axioms.

It is interesting to note that, although there is some degree of consensus that nature is random, there is also a general opinion among physicists that they have freewill, which is neither deterministic nor random. The two views are at odds with one another, although it is possible to invent creative solutions \cite{mckenna2004compatibilism}. A small minority contend that freewill is not real and that even consciousness does not exist. We will not explore that view here.

Consider the following thought experiment. We start with a universe without freewill animals living in it and that is ideally random. We assume that freewill actors, as self-organized particle systems (e.g., humans), “contaminate” this otherwise perfectly random system with their non-randomness - their freewill. Accordingly, they “steer” or causally influence the particles of their bodies by their creative and strategic freewill choices of thoughts and actions, imparting non-random order on the spacetime and particles in the rest of the universe via gravitational, electromagnetic, quantum entanglement and quantum wave function resonance and damping interactions. This ordering influence is ubiquitous because there is no cutoff on the range of force interactions and because each influenced particle in turn influences others. The \textit{free will theorem} and the \textit{strong free will theorem} of John Conway and Simon Kochen states that, if we have freewill (i.e., our choices are not a function of the past), elementary particles must have some form of that same freewill quality \cite{conway2006free,conway2009strong}. That is, particles would behave neither deterministically nor randomly. Henceforth, we use the term freewill implicitly meaning Conway and Kochen's sense of freewill.

%Your text comes here. Separate text sections with
\section{Is Reality Information Theoretic?}
\label{sec:1}

An insightful pathway to explore the code theoretic axiom is to first decide on a related axiom, which can be introduced by the question: \\

\textit{Is reality made of information or merely described by information?} \\

John A. Wheeler was one of the first modern physicists to argue that nature is information theoretic \cite{wheeler1990information}. Today, there are a large number of physicists, such as Stephan Wolfram \cite{wolfram2002new}, Gerard ‘t Hooft \cite{hooft2003can}, Ed Fredkin \cite{fredkin1990digital}, Jüergen Schmidhuber \cite{schmidhuber1997computer}, Seth Lloyd \cite{lloyd1996universal}, David Deutsch \cite{deutsch1998fabric}, Paola Zizzi \cite{zizzi2000emergent},  Carl Friedrich von Weizsäcker \cite{weizsacker1948geschichte} and Max Tegmark \cite{tegmark1998theory}, who suggest it is too aggressive to theorize reality is made of something other than information. They contend it is more conservative to accept the logical indication that reality is made of information.

One of the supposed evidential highlights of the information theoretic argument was first observed by James Gates Jr . He discovered the most fundamental error correction code from computer theory, block linear self-dual error correcting code \cite{doran2008relating}, embedded in the supersymmetry equation network that unifies all fundamental particles and forces other than gravity\cite{nilles1984supersymmetry}. 

It is interesting to note that there is not a good counter argument to the information theoretic ontological axiom. Specifically, when one tries to define energy as anything other than information, they must take a Platonist view that claims energy \textit{just is} --
a sort of primordial stuff for which we have no further explanation other than knowing how it behaves. The is the Platonic view that says nature is made of some fundamental stuff for which there is no explanation for how it comes to be. Similarly, this end-of-the-road statement that energy \textit{just is} is identical to the information theoretic ontology if one stops at the axiom that information \textit{just is} without going further to explain how this information comes to exist or what it is made of. 

To simplify, if energy is not information but is the ultimate stuff that \textit{just is}, we know (1) how it behaves but we do not know what it is made of or how it came to be. On the other hand, if we say energy is made of information, then we know (1) how it behaves and (2) what it is. But we do not know what information is made of or how it came to be. \\

\textbf{Information is:} \textit{Meaning conveyed by symbolism.}\\

The increasingly popular view that “energy is made of information” goes one step further into clarity and explanation than saying “energy just is" \cite{rovelli2015relative}. But it does not go far enough. An understanding of what information is made of is lacking. How does symbolic meaning -- information -- come to exist? It is not necessary to stop the scientific inquiry at a premature axiom of “information just is”. Stopping at that axiom or not is an important decision, since it would serve as the most foundational scientific axiom underlying all of physics.

\section{Reality as a Simulation}
\label{sec:2}

Some who think reality is made of information suggest it is a simulation \cite{bostrom2003we}. This is known as \textit{the simulation hypothesis}. Like in the movie \textit{The Matrix}, where a quasi-physical reality exists as an information-space, one can imagine the universe being a simulation in some large quantum computer. This recently popular view does provide an explanation that goes beyond the axiom information \textit{just is}. However, in some sense, this view is still the antiquated ontology of materialism because it  presumes our universe is made of information and that there is some outside universe that is the real non-information theoretic reality.

A better alternative is to reject the idea of an outside computer and consider a self-organized-simulation, where the symbolic code is simultaneously the hardware, software and the output -- the simulation \cite{aschheim2011hacking}. There may be a more appropriate analogy than these 20\textsuperscript{th} century computer theory terms. For example, the concept of neural networks is more physically realistic because they self-organize in nature \cite{lettvin1959frog}. They are exceedingly efficient at computing, due to their massively distributed non-local architecture \cite{pizzi2004nonlocal}. The idea of a mind-like neural network as the basis of an information only reality is interesting. Here, the neural network can be made of symbolic geometric code in a graph theoretic architecture operating on a point array in a symmetry space. The information of this symbolic system would live in the emergent pan-consciousness that evolves from the evolution of this physical code. So the code exists or lives within the evolutionary emergent consciousness, which is self-actualized and emerges from the code. The logic of this non-linear causality is explained in Figure \ref{fig:2}. Later, we will explore the physical plausibility of an emergent pan-consciousness.

% For two-column wide figures use {figure*} instead of {figure}
\begin{figure}[h!]
% Use the relevant command to insert your figure file.
% For example, with the graphicx package use
  \includegraphics[width=0.5\textwidth]{loop2.PNG}
% figure caption is below the figure
\caption{Here we show the self-organized emergence of all aspects in a Code Theoretic universe.}
\label{fig:2}       % Give a unique label
\end{figure}
%


The key idea for now is to establish the explanatory power of this view, which goes further than the previously mentioned \textit{just is} axioms. Here we would have a connected loop of explanations for a physical ontology that gives understanding of (1) how energy as information behaves, (2) what it is made of (abstract code objects in a pan-consciousness) and (3) how the pan-consciousness itself came to exist. There is a logically consistent and self-embedded causality chain that is less “faith based” than stopping at the axiom that energy \textit{just is} or information \textit{just is}.

Scientists do not agree how consciousness emerges in neural networks \cite{seager2016theories}. However, the theoretical and experimental work continues to improve in this area. Scientists discussing the simulation hypothesis \cite{bostrom2009simulation} are pushing the boundary of understanding in a positive way because they are resisting the status-quo view to accept the energy \textit{just is} axiom.

Although the information \textit{just is} axiom is arguably simpler and more logical than the energy \textit{just is} axiom, it comes with a price, which is implied in the definition of information as meaning conveyed by symbolism. That is, meaning is a quality deeply related to entities capable ascribing or actualizing meaning. This can be solved by saying that we live in a simulation of aliens in another universe, who in turn live in a simulation of aliens in another word, \textit{ad infinitum}. If we do not accept the simulation hypothesis, due in part to this \textit{Russian-doll} problem, the information \textit{just is} axiom demands a boldly different worldview than the materialistic philosophy of energy \textit{just is}. Materialists can say that God made the energy or the big bang spewed it out. However, scientists contending that reality is information must deal with the fact that information relates to meaning and meaning relates to choice and consciousness.

According to the code theoretic axiom, the information view means that everything is information --
including the abstract neural network based code theoretic substrate itself. As long as there are physically realistic syntactical rules guiding how an abstract code self-organizes, it is equally as logical for information to behave physically as it is for the more enigmatic notion of energy as something other than information to behave physically. In this case, the term \textit{simulation} would be confusing because that word is used to distinguish between something real as opposed to something not real. For example, if dreams are unreal and waking reality is real, then we can call the dreams simulations of the real world. However, if reality is information theoretic, the terms “physical" versus "abstract” and “reality" versus "simulation” must be replaced. We may use terms related to neural networks and emergent consciousness. For example, we might say something is either “chosen" or "not chosen” or “thought" or "not thought”. This fundamental action would be identical to the idea of “observe" or "not observe” and “measure" or "not measure”. However, in the code theoretic framework, the idea of syntax comes into play, where the most fundamental freewill action is the expression of syntactically free steps in the physical code of reality. The chooser in the code, then, can logically (even if some say improbably) be an emergent pan-consciousness as well as emergent sub-systems, such as humans.

The scientific deduction that the most fundamental stuff of reality is consciousness is not new.

Werner Heisenberg \cite{ledewitz2011church} said:

\begin{quote}
    Was [is] it utterly absurd to seek behind the ordering structures of this world a “consciousness” whose “intentions” were these very structures?
  \source{    Werner Heisenberg  }
\end{quote}
Frank Wilczek \cite{wilczek2006fantastic} said:

\begin{quote}
The relevant literature [on the meaning of quantum theory] is famously contentious and obscure. I believe it will remain so until someone constructs, within the formalism of quantum mechanics, an “observer”, that is, a model entity whose states correspond to a recognizable caricature of conscious awareness.
  \source{ Frank Wilczek }
\end{quote}
Andrei Linde \cite{linder1990particle} , co-pioneer of inflationary big bang theory, said:

\begin{quote}
Will it not turn out, with the further development of science, that the study of the universe and the study of consciousness will be inseparably linked, and that ultimate progress in the one will be impossible without progress in the other?
  \source{ Andre Linde }
\end{quote}
John A. Wheeler \cite{wheeler1990information} said:

\begin{quote}
...the physical world has at bottom -- a very deep bottom, in most instances -- an immaterial source and explanation; that which we call reality arises in the last analysis from the posing of yes-or-no questions... all things physical are information-theoretic in origin and that this is a participatory universe.
  \source{ John Wheeler }
\end{quote}
How can this idea of a code and a pan-consciousness be made concrete and mathematical such that we can use it to do realistic physics? To start with, the code would need to use virtually non-subjective symbols that are quasi-physical.

\section{Quasi-physical Symbolism}
\label{sec:3}

Again, our definition of information is \textit{meaning conveyed by symbolism}. And expressions of code or language are strings of symbols allowed by syntax -- ordering rules with syntactical freedom. A symbol is an object that represents itself or another object. And an object is anything which can be thought of. In the universe of all symbols, there is a special class with very low subjectivity. They can be called self-referential geometric symbols. For example, we can represent the meaning of a square with the Latin letters “square”. Or we can represent it with the symbol of a square itself, in which case it is a self-referential symbol. Quasicrystals, such as the Penrose tiling \cite{penrose1974r}, are examples of geometric symbolic codes. Geometric codes are defined as a finite set of geometric “letters” or shapes and ordering rules with syntactical freedom. Because the universe is geometric and in 3-space, the logical symbols of an underlying code would be polyhedra. Both the ordering rules and dynamic rules should be based on geometric first principles, as opposed to arbitrary or invented rules. \textit{In Toward the Unification of Physics and Number Theory} \cite{irwin2017toward}, we showed how shape numbers, as geometric symbols for integers, are uniquely powerful. \textit{In A New Approach to the Hard Problem of Consciousness}, we elaborate on quasicrystalline codes as a logical basis for a quantum gravity framework \cite{irwin2014new}.

The standard model of particle physics is generally considered to be the most powerful physical model we have \cite{oerter2006theory}. It synthesizes quantum mechanics with particle collider data to show how all known fundamental particles and forces (other than gravity) are gauge symmetry unified according to special algebraic and group theoretic structures corresponding to higher dimensional polytopes and lattices. A quasicrystal is an irrational projection to a dimension $n-m$ of a slice of an $n-$dimensional lattice. The projection preserves (under transformation) key information about the higher dimensional lattice and its associated Lie algebra. For example, a 3D quasicrystal derived from the $E_{8}$ lattice encodes the gauge symmetry unification of the standard model of particle physics insofar as $E_{8}$ and any of its subspaces, such as $E_{6}$, encodes such unification physics \cite{gursey1976universal}.

We are aware of only one class of non-invented codes that exists via first principles within the universe of all codes both geometric and non-geometric. That class of codes is the set of all quasicrystals. Each is generated by an irrational projection of a lattice slice to a lower dimension \cite{jaric2012introduction}. The standard model of particle physics and associated gauge symmetry models correspond to Lie algebras and associated lattices \cite{becker2006string}. The lattice analogue starts with the idea that different particles and forces are all equally related to the homogeneously arrayed vertexes or root vectors of certain hyper-lattices, such as $E_{8}$. In order to make such models physically realistic and dynamic (asymmetric), various symmetry breaking mechanisms have been proposed. There is poor consensus on what this mechanism is because none are very convincing \cite{weinberg1992dream}.

We propose that projective geometry may relate to the correct mechanism. Conveniently, this generates (1) the only known non-invented and first principles based codes and (2) an elegant first principles based symmetry breaking mechanism. Because the projection is irrational, it preserves under transformation the necessary gauge symmetry unification physics.

One of the most important of the 19 parameters of the standard model of particle physics, the Cabbibo angle, can be written in the form $\cos^{-1} \left (\frac{\phi^2}{\sqrt{2(\phi +2))}}  \right )^{\circ} $ \cite{kajiyama2007golden,king2013neutrino,king2012tri,minakata2004neutrino}. It corresponds to particle collider experiment scattering angles. As mentioned, various methods have been proposed to explain how it is that reality is not symmetric and why particles are not unified but different, while possessing unification gauge symmetry values corresponding to higher dimensional lattices.

Interestingly, the angle necessary to break the symmetry of $E_{8}$ and create the 4D Elser-Sloan quasicrystal in $H_{4}$ (the only possible quasicrystal derived from $E_{8}$ that possesses $H_{4}$ symmetry) and the 3D quasicrystalline spin network in $H_{3}$ that we work with is this same angle, $\cos^{-1} \left (\frac{\phi^2}{\sqrt{2(\phi +2))}}  \right )^{\circ} $ \cite{fang2013cabinet}.

Accordingly, we contend that particle collider data and the standard model itself are evidences that irrational projection from $E_{8}$ to lower dimension correlates to the correct symmetry breaking mechanism. As stated, this generates a geometric code of spacetime and particles -- specifically a dynamical quasicrystal code. This code, like many codes, may require an error detection and correction mechanism. Quasicrystals naturally correspond to powerful error correction and detection mechanisms, such Fibonacci error correction code  \cite{esmaeili2010fibonacci}. 

%\section{Problems with the Axiom of Randomness}
%\label{sec:4}

%Unprovable axioms lie at the heart of all physical theories and mathematical proofs. A good axiom is based on logic and common sense. It should also be supported by physical evidence when possible. The problem with the assumption of randomness is that there is no decent physical evidence. The previous sentence may seem controversial, however, we think it is warranted for a few reasons: Recent results \cite{pironio2010random} do claim to prove the existence of genuine randomness in quantum mechanical systems. However, the authors neglect to clarify that freewill at the particle level, as discussed by Conway and Kochen, can equally explain the  we argue that freewill of the system and its components could yield the same experimental results. 

%we argue that freewill of the system and its components could yield the same results. The difference between free will choices and random ones is this; once hijacked by sentient creatures (such as humans and other animals), the previously random choice becomes targeted in a sense, with an element of strategic planning being introduced. A phenomena we arguably do see in the world around us and that is perhaps analogous to what Conway and Kochen refer to as "semi-freedom" \cite{conway2006free,conway2009strong}. We also see that creatures with higher consciousness are able to affect their environment (steer the previously random choices) to a greater degree than lesser consciousnesses.

%One can consider that a human with supposed freewill influences the randomness of fundamental particles in her body. For example, a dancer moving in an orderly freestyle performance uses freewill to steer the vectors and dipole orientations of all fundamental particles in her body according to her freewill. The quantum wave functions and gravitational connections between her creatively influenced particles and every other particle in the universe is thus something that must be accounted for. In some sense, she has “polluted” the entire universe with her creatively or strategically controlled causal connections, both classically and quantum mechanically. If the universe without her could be imagined to be purely random, then she has contaminated it through and through with her freewill.

%There is also no theory for the generator of randomness. It is a belief without explanation or proof - an axiom. The theory of randomness is essentially the idea of order versus non-order in the universe. An example of disorder relates to the theory that spacetime is smooth - without substructure, such that particles have no restrictions of when and where they can exist. However, space and time may be quantized, like a pixilated video monitor. There is no consensus yet on this question. If it is not quantized, we can imagine that a particle in, for example, the Dirac sea of virtual particles can appear for a moment at any possible coordinate on a line between locations A and B and at any moment in time between moments A and B. In other words, the particle is not constrained by any ordering rules that prevent it from appearing at a certain time or at a certain location in space. This idea of smoothness or lack of constraint is what many physicists presume to be true because no one has put forth a predictive quantum gravity theory. As a result, the popular default is the older presumption that space and time are smooth. Logically, however, there is no reason this should be the default idea, as opposed to the presumption that it is pixilated - discrete just as smooth looking water is actually made of discrete molecular building blocks. There was no evidence against the molecular structure of water. So the default presumption could have logically been that it was made of particles, even before we had equipment capable of detecting them.

%A code is the quintessential system of order. For example, codes must have a finite set of symbols and limited syntactical or ordering rules. The finiteness is a restriction of freedom. One could say that, if nature is a geometric code in action, she expresses herself as restricted spatiotemporal coordinate changes of particles. Such a fundamental particle code would impose syntactical restrictions on where and when a particle can appear in the Dirac Sea within spacetime. The argument for quantized spacetime in unification models has been well developed by various scientists working on quantum gravity theories as well as by quantum mechanical theorists, who postulate that a length in space can be no shorter than the Planck length and a duration can be no less than the Planck time.

\section{Challenges with the Code Theoretic Axiom}
\label{sec:5}

As discussed, the challenge with the code theoretic axiom lies in the fact that geometric symbolism requires some notion of consciousness to actualize the information or meaning into existence. Of course, one may take a Platonist philosophy and suggest that these abstract or quasi-physical geometric symbols that constitute reality simply exist without need of an actualizing entity for the symbolic meaning. In other words, one can decide that the Platonic realism of the symbols is itself the ground of reality and the unprovable axiom -- they just \textit{are}. As mentioned, that is identical to the decision to accept that energy \textit{just is} without further explanation.

So the first challenge is the fact that information is meaning and meaning requires consciousness to actualize or recognize it. Certainly, animals such as humans are not likely to be the actualizers of all of this microscopic meaning. The second challenge with the code theoretic idea is the issue of syntax choice. What chooses the syntactically free steps in the code? If nature were just a deterministic causal algorithm playing itself out, one could say energy is like a set of falling dominoes with no need for a chooser because there are no syntactically free steps in the algorithm because it is not a code. One could then stop at energy in the search for further explanations. But with a code, stopping at the axiom that energy is information and information \textit{just is} is not so easy because of this issue of the need for a chooser of the syntactical freedom. If one introduces randomness as the syntax chooser, it is problematic because the meaning output of a code degrades when randomness is introduced. As an analogy, we can take a paragraph of a book to see what happens with the quality of meaning of the English code output when we replace each adjective and noun with randomly chosen adjectives and nouns. The syntax rules will be legally followed but the code conveys much less meaning.

Furthermore, even if we decide that code efficiency is not important and say the ultimate \textit{stuff} of reality is randomness or an unexplainable quantity called energy that randomly operates the code syntax, it would be just one of the other axioms that stop at some level without further explanation -- accepting on faith something to be the true base of physical reality, even though there is no explanation for it.

Careful reasoning gives us logical permission to consider that the definition of information as \textit{meaning conveyed by symbolism} should be taken seriously. That is, if reality is information and code theoretic, meaning must be involved. And meaning is a substance of minds. A mind chooses (observes) or actualizes information -- creates meaning.

So where do we go from there in truly critical scientific inquiry without romantic or spiritual motivations but with only logic and reason to guide us? Are we to seriously consider this notion of pan-consciousness as the substrate of reality -- this idea of a \textit{Star Wars} type intelligent \textit{Force} or some other fictional or religious sounding notion?

There is a rigorously logical possibility with physical evidence that is no less remarkable than big bang theory or the fact that human consciousness emerged from quarks and electrons. And it is certainly less fantastical than the idea that we are living in a simulation of some other real universe.

\section{The Possible Origin of Pan-consciousness}
\label{sec:6}

Consider the non-linear physical logic that event A causes B, which causes C, which completes the loop by causing event A. Many scientific works have put forth theoretical and experimental evidence for retro-causal feedback loops \cite{sarfatti2011retrocausality}. Dean Radin has done experiments showing retro-causality in the form of human skin conductance changes correlated to computer monitor displayed images not yet selected by a random number generator \cite{radin_electrodermal_2004}. The delayed choice quantum eraser experiment has shown how the freewill choice of an experimentalist changes events in the past. Wheeler argued how such choices loop back to retro-causally influence things billions of years ago. Susskind and Maldacena argue that the wormholes or Einstein-Rosen bridges linking non-local regions of spacetime predicted by general relativity are equivalent to quantum entangled particles predicted by quantum mechanics \cite{susskind2016addendum,maldacena2016black}. And in 2012, physicists in Israel experimentally demonstrated that particles can be entangled to influence one another over time \cite{brierley2015nonclassicality}.

There is no evidence retro-causality is unrealistic. There is some evidence suggesting it is a real phenomenon. And there are strong theoretical implications in both general relativity and quantum mechanics that non-locality is a deep aspect of reality. Furthermore, there is no predictive quantum gravity theory of spacetime that includes particle physics whereupon one can make strong statements about what should and should not be possible with respect to retro-causality and non-local connectivity.


%\section{Challenge with the Code Theoretic Axiom}
%\label{sec:7}

We suggest that if it is possible for human consciousness to emerge from finite quantities of energy within some non-local quantum gravity framework, that it is either inevitable or possible that consciousness eventually emerges from all quantities of energy in the universe.

This is an outrageous idea that deserves careful critical thinking. We will deduce via asking and answering a few questions below. Before we begin, the objective here is to look for a logically consistent explanation for how the universe can be self-actualized --
a self-emergent neural network that is its own hardware, software and simulation output as one in the same system. We are looking to see if consciousness itself can be the most physically realistic and plausible axiom instead of the imagination that we live in a computer simulation or that energy or abstract geometric symbols \textit{just are} without deeper explanation. We wish to reduce the axiom down to the irreducible idea of Descartes, rephrased here as:

\begin{quote}
I don’t know what consciousness is, but I know it exists because I have evidence --
namely the fact that I am freely choosing to wonder about what consciousness is. And that free choice, neither forced upon me by causality nor merely accidental, is part of the very definition of my consciousness.
  \source{  }
\end{quote}
Note that some define consciousness as simply being aware, and that it does not require freewill. Others define consciousness as correlating to freewill and choosing what to be aware of or to observe. Notice that “to be aware” is a slippery notion. What does it mean? Does it mean you are receiving information about something? That would not suffice, since we are constantly receiving information of which we are not aware. So if we adopt the definition of information that does not require freewill or the choice of what to be aware of, we have a blurry enigmatic and therefore imprecise concept of what “to be aware” means.

Conversely, to choose or select is a precise concept. A random action can select A or B. A chooser can select A or B. Embedded deep in the concepts of quantum mechanics is the notion of choice of observation/measurement --
position or momentum, for example.

Furthermore, the idea of consciousness being defined only by awareness and stripped of freewill or choice of what to be aware of is the notion of something outside controlling your mind to insert only the thoughts to be aware of that it chooses. In this sense, you would be a copy of their consciousness or its consciousness --
merely the river of thoughts defining something or someone else, like a clone of their experience.

Accordingly, we adopt here the more precise concept of choice and freedom as the defining quality of consciousness/awareness. Of course, this is a convenient quality for a code theoretic ontology which requires a chooser at the syntactically free steps in a quantum gravity physical code of reality.

We define consciousness as:
\begin{quote}
Something capable of making non-random and non-deterministic selections --
choices --
something which can actualize or choose a meaning to recognize.
  \source{  }
\end{quote}
The reason for the second sentence is because, if one has freedom to choose a selection, by definition, one has freedom to choose something to observe, measure, be aware or think of. They have the freedom and the ability to actualize meaning.

The following series of five deductions helps connect some of the elements of this \textit{lesser of evils} approach to the question of the ultimate \textit{stuff} of reality --
the quest to find a maximally reduced axiom of realty with the highest explanatory power possible.

\paragraph{\textbf{Deductive Question 1:}}

\textit{Does consciousness exist in the universe?}\\

Descartes simplified things nicely by supposing that because he questions whether he exists, he must exist. He said, “I think, therefore I am”. He was not specifically speaking of his physical body. It was his inquiry as to whether or not he, as in his self or consciousness, exists.

The answer seems to be, yes we are conscious because we can choose what to think about --
what to be aware of and what meaning to give it.
\paragraph{\textbf{Deductive Question 2:}}

\textit{Does physics place an upper limit on what percentage of the universe’s energy can self-organize into conscious systems or into a network of conscious systems that is itself conscious?}\\

To think about this question, let us imagine we are examining the universe 4-billion years ago. We are considering single celled organisms and agreeing that we cannot predict their primitive choices of action and behavior. We label them with some primitive notion of freewill and awareness of their environment and their boundary --
their selves. We do not have to admit they have the ego based questions about self that we do. But they do have a sense of their environment, internal structure and the boundary between the two. They chase food, run from predators, reproduce, excrete waste and are absolutely unpredictable in their primitive choices. So, 4-billion years ago, we debate whether or not larger magnitudes of energy can self-organize into more highly conscious systems and whether or not the single celled organisms can self-organize into systems that are of a higher rank of complexity and consciousness. Zooming forward to today, we find that self-organization turns out to allow about 37 trillion single celled organisms to become the emergent consciousness of a human mind-body system. Clearly, there appears to be no law of physics that would prevent a more sophisticated consciousness than a human to self-organize in the universe. There also seems to be nothing to prevent multiple human consciousnesses from knowingly or unknowingly being part of an uber-consciousness similar to how many single celled animals self-organize into a larger smarter system like a human without fully eliminating their primitive individual freewill. Notwithstanding classic physical arguments, the only logical or conservative upper limit would be all the energy in the universe in terms of what percentage can self-organize into a system of conscious systems that is itself conscious. There is a mathematical and physical idea that some consider provable:

\begin{quote}
Given enough time, whatever can happen will happen.
\source{}
\end{quote}

Based on these carefully reasoned ideas, we may simply say that somewhere forward of us in spacetime, a universal scale consciousness or global network of consciousnesses that is itself conscious has emerged. One cannot use the separate regimes of quantum mechanics or general relativity to argue for or against this notion. For example, one could use general relativity in a naive attempt to suggest networks across spacetime may not perform well because of the limitation of the speed of light. This does not hold well in light of the experimental and theoretical evidence discussed above. The fact is that, without a predictive quantum gravity theory, the two separate place holder models of general relativity and quantum mechanics are incomplete pictures of physical reality and cannot give us an answer as to what is or is not possible with respect to trans-spatial and trans-temporal networks, especially when correlation between two or more nodes in such networks do not exchange information at a finite speed. 

Accordingly, the answer to this deductive question \#2 here is:

\begin{quote}Theoretically, all the energy in the universe can self-organize into a conscious system. And because it is possible, it may be exceedingly probable that at some point ahead of us in spacetime, it has occurred.
\source{}
\end{quote}

The interesting thing about this deduction is not that it must be correct. It is simply noteworthy as a contrast to other axioms such as randomness, which have very little logical or deductive evidence. Logical evidence is clearly not proof. And of course, there are no proofs in physics at all. But the logical consistency of the idea and the explanatory power is perhaps more scientific than the dead-halt at the unsupported axiom of randomness. It is the lesser of evils. At very least, it is more congruent with the scientific process of searching to go beyond axioms and into deeper and evermore logical explanations.


\paragraph{\textbf{Deductive Question 3}}

\textit{How many times would the human population have to double to require every atom in the universe to be part of a mind like system?}\hfill\break

The answer is about 70 doublings. This can happen over thousands of years, as opposed to geological or cosmic time scales. Obviously, resource limitation always halts doubling algorithms in nature, whether that be a bacterium doubling on the surface of an apple or a population of rabbits doubling on an island. An intelligent animal population would have to develop technology to move beyond their biosphere and into the universe at large in order to avoid resource limitations that would prematurely halt the doubling algorithm before all energy in the universe could be converted into a network of conscious systems that is itself conscious. With global violence and pollution decreasing every year for the last 50 years straight and with technology doubling at an even faster rate than the decrease in violence, the probability that humans will move out beyond Earth’s biosphere has never been as probable as it is today.

\paragraph{\textbf{Deductive Question 4}}

\textit{Is there any law of physics that would prevent consciousness from self-organizing in the electromagnetic spectrum of space or within emergent patterns of quasiparticles?}\\

There is no known prohibition by physical laws. In fact, a series of recent breakthroughs in the manipulation of bosons has occurred. We can now completely stop light inside a crystal \cite{tsakmakidis2014completely}. We can tie light into knots and braids. Again, the question ultimately depends upon the nature of a possible substructure of spacetime. Our group’s quantum gravity and particle unification formalism is called \textit{emergence theory}. The base mathematical object is a quasicrystalline array of points in $H_{3}$ symmetry space assumed to be the Planck scale substructure of spacetime. It is called the \textit{quasicrystalline spin network} \cite{fang2015icosahedral}. When acted upon by a certain binary geometric code of “on/off” connections, it acts as a neural network mathematically. The fundamental propagators or quantum particles are called phason quasiparticles, and they are inherently non-local. So we have no theory for how consciousness could emerge from our formalism. However, we do have knowledge of how particle physics and gravity theory can emerge from it. And, of course, it seems true that human consciousness emerged from the self-organization of spacetime and particles -- physics that is.

Harald Atmanspacher explains consciousness using quantum field theory\cite{atmanspacher2004quantum}. He says, since quantum theory is the most fundamental theory of matter currently available, it is a legitimate question to ask whether it can help explain consciousness. Large systems have less symmetry than nearly idealized microscopic systems. Jeffrey Goldstone proved that, where symmetry is broken, Nambu-Goldstone bosons are observed in the spectrum of possible states; one canonical example being the phonon in a crystal \cite{goldstone1962broken}. A phonon is a quasiparticle similar to the idea of a phason quasiparticle propagating in a quasicrystal. Ricciardi and Umezawa proposed a general theory of quanta of long-range coherent quasiparticle waves within and between brain cells \cite{ricciardi1967brain}. They showed a possible mechanism of memory storage and retrieval in terms of Nambu-Goldstone bosons. This was later advanced into a theory including all biological cells in the quantum biodynamics of Del Giudice. Mari Jibu and Kunio Yasue later popularized these results with respect to consciousness theory \cite{jibu1995quantum}. Susan Pockett and Johnjoe McFadden have proposed electromagnetic field theories of consciousness \cite{mcfadden2002conscious}.

The point of deduction \#4 here is simply to be aware that there is no strong logical or scientific evidence to suggest consciousness can only exist in atomic or fermionic states of energy. There are only hand-waving theories that can be conjured for why this is impossible, just as there could have been theories by some hypothetical arguers 4-billion years ago as to why a Wi-Fi signal broadcasting the Internet cannot emerge from a single celled organism. Of course this turned out to indeed be possible.


\paragraph{\textbf{Deductive Question 5}}

\textit{If the universe is expanding at faster than the speed of light or at the speed of light, how could consciousness that escapes a biosphere ever move out into all of the universe to create a network of consciousnesses that is itself a higher order of consciousness -- a universal consciousness acting as the substrate of an information only universe?}\hfill\break

The full exploitation of wormholes predicted by general relativity and non-local connections over time and space predicted by quantum mechanics (and experimentally demonstrated) is not likely to be possible without a predictive quantum gravity theory. The conservative scientist should simply be leery of naive claims of impossibility based on general relativity or quantum mechanics alone because the relationship of the two frameworks includes serious conceptual contradictions of one another. Also, both of these place-holder theories will someday be improved upon by a predictive quantum gravity theory that will show how the assumptions of each may be partially flawed or do not apply in special cases, although most aspects of each theory should hold true.

\section{An Insignificant Force Emerging to Become Everything}
\label{sec:9}
An interesting analogy is a few million bacteria on an apple. Intermolecular forces, gravity, the environment, etc., all define the form and behavior of the apple. However, after only a few doublings, the bacteria overtake other factors to become the primary influence determining the destiny of the apple, breaking molecular bonds to return the elements back to the soil. The universe is not old. It is just getting started. An average sized star, such as our sun, lives for about 10 billion years. This means that, from our vantage point “back here” on 21st century Earth, the universe is barely 1.5 solar lifetimes or generations old. Like the very beginning of the bacterial doubling algorithm, from this early stage, it appears that consciousness is a trivial influence existing in the tiniest fraction of the overall energy -- merely along for the ride while the ordinary physical forces determine everything. However, if a doubling algorithm gets started by a species that has escaped its biosphere and which has discovered a non-local quantum gravity theory, and technologies derived therefrom, trans-temporal forms of consciousness could emerge. In this case, it would not be illogical to entertain the possibility that this “supernovae” of exponentially exploding consciousness defines the future of the universe from our vantage point and is the irreducible foundation of the universe when spacetime as a whole is considered. We might even go so far as to conjecture that this might tie into the observed acceleration of the rate of expansion of the universe. That is, exponential algorithms on increasingly connected networks have an exponential growth curve, wherein the rate of exponential growth itself exponentially increases.

\section{The Non-computable Substance of Reality}
\label{sec:10}
So we have arrived at a seemingly mystical and yet somehow logical and explanatory axiom that the ground of reality is consciousness -- an implication of the code theoretic axiom. It is worthwhile to discuss one important mathematical aspect of this substance. Let us introduce the idea with a surprising party trick. Imagine selecting 17 people from a birthday party and putting them in a room to vote on how many combinations they can form from members of their small group. For example, there can be Linda and Sam and there can be Sam and Linda. There can be Sam, Linda and Gary and there can be Gary, Linda and Sam. We can combine the names and the ordering of the names. Most people unfamiliar with the math would not guess that it is over 355 trillion permutations or about 50,000 times the entire human population. A system of 17 electrons has far more interaction complexity than this, as they interact in various combinations of quantum wave function resonance and damping values and gravitational relationship states. A single human brain has over 100 billion neurons. And each neuron has over 100 trillion atoms, which each contain a quantity of fundamental particles. These interactions, which humans still only partially understand from the equations of the two incomplete pictures of reality, general relativity and quantum mechanics, are the actual physical substance and behavior of reality. The still mysterious and debated ontological nature of the quantum wave function is, in part, the probability space object arrayed in 3-space that partially describes these non-computable interactions.

The emergence of physics and our reality comes from the non-computability of these interactions. That is, they are non-computable in a finite universe, even in principle, and yet they not only exist -- they are the most realistic substance of reality itself. Why non-computable? Consider that we live in a finite universe of a finite age. If a computer were made from all the energy in the universe and given, say, 100 trillion times the current age of the universe to compute the interactions of the particles of just one brain cell, it would not be remotely possible. And yet, actual reality is the emergent result of the oscillators in that one cell interacting with all other oscillators in the universe. And below that level, there may exist a theoretical Planck scale graph theoretic substructure contemplated in approaches such as ours or loop quantum gravity. The idea is that whatever this substance of “consciousness” as the ground of reality is like, it is non-computable, even in principle. And yet, it is perhaps the most real and foundational stuff of reality.

\section{Examples of Possible Predictions Indicated by the Code Theoretic Axiom}
\label{sec:11}

Physical ontology is what science is about. Ontology is the study or labeling what is real and what us unreal. Physics is the study of better modeling what is known to be real and discovering new phenomena that are real. Sometimes the models predict things that are not observed at the time, such as black holes or the molecular substructure of water. The code theoretic axiom can inspire scientific predictions. For example, when a physically realistic quantum gravity code theoretic framework is discovered, it will...

\begin{enumerate}
  \item ...be based on an error correction and detection scheme.\\
  \item ...lead to the \textit{principle of efficient language} (PEL), which will demand that the universe operate as a relationship between $E_{8}$, $H_{4}$ and $H_{3}$.\\
  \item ...because of the PEL, have as its numerical basis the Dirichlet integers 1 and the inverse of the golden ratio. Dirichlet integers have unique properties which make them suitable for the generation of optimal codes. For example, they are a closed Euclidean ring of quadratic integers, they are dense in the real numbers, and possess a unique prime decomposition. They are deeply related to the Fibonacci sequence by their algebraic units which involve Fibonacci numbers. And they are powers of the golden ratio. This links them fundamentally to specific error correction codes, like Fibonacci error correction and detection codes \cite{fraenkel1996robust}.\\
  \item ...use the angle $\cos^{-1} \left (\frac{\phi^2}{\sqrt{2(\phi +2))}}  \right )^{\circ} $, which is the scattering angle relationship between fundamental particles according to certain particle mixing matrices\cite{kim2011quark}. This is because in order to generate the densest network of Fibonacci chains in any dimension, one must project a slice of the $E_{8}$ lattice to 4D along this angle. And this angle too must exist in the 3D space where graph theoretic formalism would express its dynamical selection patterns.\\
  \item ...because of the PEL, operate in a binary point space as a neural network formalism that exploits the two densest possible networks of Fibonacci chains in any dimension (the quasicrystalline spin network and the Elser-Sloan $E_{8}$ to 4D quasicrystal).\\
  \item ...because of the PEL, use a physical possibility space in 3D that is the quasicrystalline spin network due to a secondary binary code allowable in 3D that is related to chirality and periodicity \cite{sadler2013periodic}.\\
  \item ...involve an interaction with emergent consciousnesses, such as humans, that actualize \textit{class II meaning} (see addendum) with respect to experiments such as the double slit experiment. For example, it will cause a change in the interference pattern if a human can in the future or present use the position information of the measurement to actualize class II meaning. This will break the symmetry of distribution of frames relative to a formerly equal treatment of the two slits.\\

\end{enumerate}

\section{Conclusion}
\label{sec:12}

The deductive thoughts above are a string of carefully reasoned choices about what might be more likely than not. Via this deductive approach, which rejects aggressive or non-maximally reduced axioms, we land on the ultimate axiom. Consciousness exists because we are choosing to wonder if we are conscious. And because we have evidence that our consciousness exists, the argument that consciousness is the foundational substance is better justified than speculations with less evidence, such as the simulation hypothesis. It is also more explanatory than stopping at the axiom that energy \textit{just is} or that some abstract information theoretic Platonic symbols \textit{just are}.

Axioms are always “religious” in some sense, where that term implies faith or belief in something that cannot be shown to be true. However, good axioms are carefully reasoned. Structureless smooth spacetime is an example of a weak axiom with no reasoned logic or evidence to support it -- just as there was no good evidence supporting the belief that water is a smooth continuous substance. Resting comfortably on aggressive physical axioms, such as energy \textit{just is}, prevents exploration of further truth and leads to possibly false scientific ideas. For example, if we accept the axiom that spacetime is smooth, it becomes mathematically logical that a black hole contains an infinity at the center -- a singularity. However, if spacetime is quantized, there is no singularity. Clearly, our axioms can be dangerous if they are unsupported by experimental evidence or logical reasoning.

Penrose \cite{hameroff2014consciousness}, Tononi \cite{tononi2012integrated}, Koch \cite{koch2012consciousness}, Nagel \cite{nagel2012mortal}, Dretske \cite{dretske1997naturalizing} and many others have written about the notion of a pan-consciousness being physically realistic and logically necessary. The plausible theory of a pan-consciousness as the substrate for a code-theoretic physical framework is more natural and less fantastical than the popular idea growing in academic circles that the universe is a computer simulation existing in a different universe. It is more realistic because we have physical evidence for the sub-parts of the idea: (1) Consciousness self-organizes from fundamental particles and forces, (2) there is no upper limit on how sophisticated it can become or how much of the energy of the universe can self-organize into it and (3) neural network formalism, not computer theoretic formalism, is where and how consciousness emerges physically. Neural networks operate according to codes, not deterministic algorithms.

The plausibility of all energy self-organizing into a conscious system is not logically problematic, given what we know of physics today. What is problematic is the idea of a trans-temporal consciousness and retro-causality, which one would presume is necessary to act as a substrate for the physics of spacetime and particles. That is the concern, not the probability of exponentially self-organizing consciousness. The lack of certainty about this lies in the fact that there is not a predictive quantum gravity theory that can predict the possibility or impossibility of trans-spatiotemporal networks. However, with the recent work of Susskind \cite{susskind2016addendum} and Maldacena \cite{maldacena2016black} and the fact that general relativity and quantum mechanics both allow non-local connections, it seems more plausible than not plausible. Accordingly, until a predictive unification theory is discovered, we can realize that there are no “deal killers” to the notion of retro-causality. Indeed, there is some physical evidence for it in the form of Dean Radin’s experiments \cite{radin_electrodermal_2004} and various delayed choice quantum eraser experiments \cite{peruzzo2012quantum}. And we do know with experimental certainty that nature is inherently non-local, where entangled particles are causally connected over arbitrarily large distances of time and space.

If the universe is code-theoretic, it traffics in the substance of all codes -- meaning. Geometric or physical meaning has virtually no subjectivity, while other forms of meaning, such as humor, are highly subjective. Similarly, geometric symbols have very low subjectivity because mathematical meaning is encoded directly into the symbols themselves\cite{irwin2017toward}. For example, the body diagonal of the self-referential symbol of a square is the length times the square root of 2 -- intrinsic meaning with very low subjectivity. Such symbols have the ability to act as the quasi-physical symbols/building-blocks of a geometric reality.

Figure \ref{fig:2} represents the loop of five causally connected phases of the code-theoretic universe. It shows the self-actualized hierarchical loop of emergence. It is approximately as fantastical as big bang cosmology and the simulation hypothesis\cite{bostrom2009simulation}. It is physically plausible and logically self-consistent. It rests on the most reduced axiom possible, the deduction of Descartes. We hold it out as the lesser of evils, where all deep fundamental physical and cosmological models are audacious but where a scientist must choose the one with the best explanatory power, logical self-consistency and most irreducible starting axioms.

\newpage
\section{Addendum: The Principle of Efficient Language}
\label{sec:13}

The code theoretic axiom leads directly to the principle of efficient language. We provide here a preview summary of it.

One might say that the overarching principle of classic physics is the principle of least action. It directly led to Noether’s second theorem about symmetries in nature, which underlies the most powerful physical theory, the standard model of particle physics. If the code theoretic axiom is true, reality is about meaning. There should, therefore, be a more general universal principle of which the principle of least action and Noether’s theorem are special cases. Put differently, those two foundational principles would be recast as predictions and manifestations of the overarching principle tied to the code theoretic axiom, the principle of efficient language, which can be defined thusly:

\begin{quote}
Because reality is code theoretic, its purpose is to efficiently express meaning with its ultimate conserved quantity -- quantized actions of the evolving pan-consciousness substrate, specifically syntactically free binary choices in the self-emergent code theoretic network. Efficiency is achieved by (1) operating as a neural network code that generates maximal meaning from binary actions and (2) strategically places these syntactically free choices in order to generate maximal physical and higher order meaning.
\source{}
\end{quote}

Following, is a summary of the ideas necessary to make sense of this otherwise obscure definition:

\subsection{Neural Network}

A neural network, as opposed to a standard computer, is an array of points distributed in space upon which information can be creatively computed and in which information is exchanged. Computer theory is concerned with efficient creation of information, the solution of problems. Information theory is concerned with efficient transportation and networking of information. Neural network theory is concerned with the efficiency of both. Nature has demonstrated that freewill can emerge in a neural network and act back upon the systems behavior in a feedback loop becoming the emergent behavior of the network \cite{aleksander1990introduction}.

\subsection{Conserved Quantum of Action}

In a physical neural network, the conserved quantity is energy, which is used to turn a connection on or off. In an abstract or information theoretic neural network living as information in a self-emergent consciousness, the fundamental binary action is a choice to recognize or register a connection between points/nodes as being on or off. Part of the mathematical formalism of such a neural network theory is graph theory expressed on a spatial graph -- a graph \textit{drawing}. If the substrate of an information theoretic reality is emergent consciousness, the ultimate conserved quantity is syntactically free choice, which is the fundamental quantum or action of consciousness.

\subsection{Quantum of Consciousness}

The simplest choice between quantities of identical things is the choice between two things. The simplest thing is either the empty set or the dimensionless point. It is difficult to build a graph theoretic neural network formalism or make geometric symbols from empty sets. Points, on the other hand, serve both purposes very well. Accordingly, one can have a possibility space of points in some symmetry space, such as $H_{3}$. When they are chosen to be “on”, they connect to other points (nodes) of the network to form geometric pattern -- the physical information of spacetime and particles.

\subsection{Efficiency}

Efficiency in this context is the greatest ratio of meaning to binary choices. We will now discuss some more concepts to help us better contextualize this definition of efficiency.

\subsubsection{Meaning}

Meaning comes in two fundamental classes: (1) Class I meaning, which is geometric and numerical. This is physical meaning with very low subjectivity. (2) Class II meaning, such as irony, appears to be transcendent of geometry and number. The substrate of reality is the point space of the network, which is inherently geometric so if irony exists in this reality it must relate to the code -- to geometry. The ordered sets of frozen states forming dynamic physics is numerical. Remarkably, class II meaning is always built upon class I meaning. For example, the thought of irony shifts particle positions and dipole orientations in the body. This then changes particles in the universe that are entangled with those particles and also changes other particles via gravitational, electromagnetic and quantum wave function damping and resonance interactions. Accordingly, the class II meaning of irony cannot transcend its connection to geometry and physicality on the network, even if one supposes it is possible for a consciousness to exist in the gravitational or electromagnetic spectrum.

\subsubsection{Conserved and Non-Conserved Quanta of Meaning}

Fundamental class I meaning is conserved and can be reduced to binary choices in the graph theoretic network. An example would be particle spin states mapping the mass of a black hole to its surface area. However, emergent class II meaning, such as the complexities of a biosphere, is not conserved. An infinite quantity of class II meaning can exist on a conserved and finite substrate of binary actions as fundamental class I meaning. This is due to a simple fact of code theory, where each emergent symbolic object in a code can act as a symbol in a higher order code. For example, 10 million letters in a novel can be scrambled. The base information of 10 million letters remains conserved. But when ordered as a code, they can form words. The words have an additional rank of meaning beyond the letters. And sentences have an additional rank of meaning above the rank of the words and so on up through paragraphs, chapters, etc. The hierarchy can continue infinitely, such that an infinite amount of higher order non-conserved meaning can emerge from a finite amount of conserved base meaning.

\subsubsection{Special Dimensions Related To Maximally Efficient Networks}

Because the principle of efficient language requires the universe to operate with the most efficient neural network formalism possible, it implies reality must be based as an interaction between 8D, 4D and 3D according to the following logic, which is only in summary form due to the limited of scope of this addendum.

\subsubsection{Two Letter Codes}

A Fibonacci chain is a 1D quasicrystal of the simplest form because it contains only two \textit{letters} or lengths. A two letter code is generally more powerful than, say, a 50 letter code. All quasicrystals are codes. The dynamism of the Fibonacci chain code is called \textit{phason dynamics}. It is rule based and defined by geometric first principles, where non-locally connected particle patterns with wavelike qualities propagate along it.

\subsubsection{Quasicrystal Code Possibility Spaces}

A Fibonacci chain can be understood as a sequence of binary operations of “on” and “off” on a point space called the possibility space, which is itself a Fibonacci chain of a smaller scale. The points that are on or off are governed by syntactical rules and degrees of freedom in the \textit{phason code}. For example, if we have an infinitely long Fibonacci chain possibility space, and we select some point to be on or off, we will force an infinite number of other points to be on and another infinity of points to be off. This is called the empire of the point that was selected to be on by the code user. Each vertex type in a quasicrystal has its own empire. The reason for this is based on the transdimensional \textit{cut + projection} geometric first principles of quasicrystals, where a shift in the \textit{cut window} in the higher dimensional lattice instantly causes many points to enter the cut window that sends points to the 1D quasicrystal and many to exit the \textit{cut window} -- making some points in the possibility space of the quasicrystal turn on and others turn off. Quasicrystals are inherently non-local, where a change at one location, influences objects at distant locations.


\subsubsection{Non-Locality of Quasicrystals}

The deep non-locality of quasicrystals makes them remarkably efficient binary codes, where a single binary choice instantly drives a large number of additional binary choices without having to exercise additional choice actions. If choice is the conserved quantity, this unique feature of quasicrystals in the universe of all binary codes makes them uniquely powerful.

With this knowledge we can now recast the question of efficiency as:

\begin{quote}
In any dimension, what is the most powerful network of Fibonacci chains, where single binary choices generate the largest quantity of automatic binary choices?
\source{}
\end{quote}

To understand this idea, consider the first example, where the registration of one point as being on in the possibility space instantly generates the automatic registration of a large number of other points as being on and off along the chain. If we crossed this 1D possibility space with another Fibonacci chain possibility space that shared the crossing point, a binary action on that point would generate twice as many automatic binary choices, generating changes on both Fibonacci chains.

We conjecture that the densest network of point sharing Fibonacci chains in any dimension is a 5-compound of the Elser-Sloan quasicrystal in 4D that we published \cite{fang2015icosahedral}, which is derived by \textit{cut + projection} of a slice of the $E_{8}$ lattice in 8D. $E_{8}$ is the densest packing of 8-spheres in 8D and corresponds to the largest exceptional Lie algebra, which generates an infinite number of higher dimensional Lie algebras. Projecting lattices of a higher dimension than that of $E_{8}$ to 4D will not generate a denser network of Fibonacci chains than the Elser-Sloan quasicrystal. No dimension higher than 4D can have quasicrystals made of Fibonacci chains. This relates to sphere kissing problems and the fact that 4D contains six regular polytopes. In all dimensions higher than 4D, there are only three regular polytopes and they are all crystal related polytopes; the hyper-cube, hyper-tetrahedron and hyper-octahedron. 3D and 4D contain quasicrystal associated regular polytopes in addition to the three mentioned above. The 4D Elser-Sloan quasicrystal has vertexes of degree 120. Our compound of five copies of this quasicrystal has vertexes with degree 600. As mentioned, quasicrystals are languages. The 5-compound of the Elser-Sloan quasicrystal has vertexes (the degree-600 vertexes) that are the convergence at a point of 300 Fibonacci chains, the highest possible in any dimension -- a conjecture we plan to prove in a future paper. The second highest density network of Fibonacci chains possible in any dimension is called the \textit{quasicrystalline spin network}, discovered by our group. It exists in 3D and is derived from the Elser-Sloan quasicrystal. Accordingly, it encodes $E_{8}$ based gauge symmetry physics. The interplay between these 4D and 3D quasicrystals is the basis of our quantum gravity program, called \textit{emergence theory}. We plan on developing a proof that the second densest network of Fibonacci chains in any dimension is the \textit{quasicrystalline spin network}.

A quantum gravity code based on the \textit{quasicrystalline spin network} would be maximally efficient in terms of the ratio of binary choices to class I and class II meaning. If consciousness, or something akin to it, is the substrate of an information and code theoretic reality, the conserved quantity would be the simplest possible choice, which is a choice of a point or a connection being on or off. And the most efficient binary choices possible in any dimension exists on the 5-compound of the Elser–Sloan quasicrystal in 4D and, in 3D, the \textit{quasicrystalline spin network}. However, our 3D object may indeed be more powerful than the compound of the Elser-Sloan quasicrystal, even though it may be the second highest density network of Fibonacci chains in any dimension. This is because it possess a second regime of binary codes based upon aperiodic patterns of alternating 3 and 5 periodicity \cite{sadler2013periodic}.  

\subsection{Code Power: Restriction of Freedom}

A powerful and general code is an ordering scheme of a small number of symbols that is maximally simple. Here we can speak of restriction of freedom. Restriction of symbol types and restriction of classes of syntax. The ultimate restriction of symbol types is 2. Anything less is not a code. And anything more, weakens the power of the code in many cases. A spatial code would be the simplest two spatial objects. Flat 1D is the simplest space and a line is the simplest dimensional object in that space. So two different lengths would be the simplest two spatial symbols, just as on and off are the simplest two symbols in a computer code.

So an important dictate of the principle of efficient language and the code theoretic axiom is that reality will use a code with the maximally reduced number of symbols and simplest syntax necessary for the simulation of physical reality. At emergent scales, such as solid state physics, the principle of efficient language predicts that when spatiotemporal freedom in a system of oscillators approaches the non-zero restriction, anomalous physics will occur. The non-zero limit is the quasicrystalline phase, where networks of atoms self-organize into 3D networks of 1D quasicrystals, which are each composed of aperiodic strings of double well potentials. That is, they organize into strings of energy wells that have a significant fraction of both occupied and unoccupied sites. This freedom at or near the non-zero limit drives high probabilities for quantum tunneling, wherein atomic coordinate changes occur with spatiotemporal coordination over long distances creating wavelike patterns in the material and exhibiting low entropy but high dynamism \cite{irwin2017cold}. This is an example of the principle of efficient language operating at a scale far larger than the Planck scale origin from which quantum gravity and particle physics emerges.

%\textbf{Conflict of Interest:} The author declares that they have no conflict of interest.



\nocite{*}
\bibliographystyle{eptcs}
\bibliography{generic}
\end{document}
